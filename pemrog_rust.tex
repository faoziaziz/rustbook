% Created 2021-08-27 Jum 21:13
% Intended LaTeX compiler: pdflatex
\documentclass[11pt]{article}
\usepackage[utf8]{inputenc}
\usepackage[T1]{fontenc}
\usepackage{graphicx}
\usepackage{grffile}
\usepackage{longtable}
\usepackage{wrapfig}
\usepackage{rotating}
\usepackage[normalem]{ulem}
\usepackage{amsmath}
\usepackage{textcomp}
\usepackage{amssymb}
\usepackage{capt-of}
\usepackage{hyperref}
\author{Aziz Faozi}
\date{\today}
\title{Pemrograman Rust}
\hypersetup{
 pdfauthor={Aziz Faozi},
 pdftitle={Pemrograman Rust},
 pdfkeywords={},
 pdfsubject={},
 pdfcreator={Emacs 26.3 (Org mode 9.1.9)}, 
 pdflang={English}}
\begin{document}

\maketitle
\tableofcontents


\section{Abstrak}
\label{sec:orge747198}
Buku ini dibuat sebagai catatan untuk rust. Rust katanya sangat bagus untuk 
system engineer, buat saya yang masih junior dalam System Engineer mempelajari
pemrograman ini menjadi cukup menarik.

\section{Install Rust}
\label{sec:org4297dec}
\subsection{Pemasangan Rust pada Linux}
\label{sec:orgfda4baf}
\begin{verbatim}
curl https://sh.rustup.rs -sSf | sh
source $HOME/.cargo/env
export PATH="$HOME/.cargo/bin:$PATH"
\end{verbatim}
\subsection{Update Rust}
\label{sec:orgf886c82}
\begin{verbatim}
rustup update
\end{verbatim}
\subsection{Uninstall Rust}
\label{sec:org2fb48f9}
\begin{verbatim}
rustup self uninstall
\end{verbatim}

\subsection{Check Version}
\label{sec:orgf5b7373}
\begin{verbatim}
rustc --version
\end{verbatim}
\subsection{Test Program}
\label{sec:org9ac4130}
\begin{verbatim}
mkdir 1_Hello_World
cd 1_Hello_World/
\end{verbatim}
Buatlah sebuah code untuk untuk melakukan test program dengan nama
\begin{verbatim}
main.rs
\end{verbatim}
Tulis kode berikut
\begin{verbatim}
fn main() {
println!("Hello, world!");
}
\end{verbatim}
Kemudian kompile dengan perintah berikut
\begin{verbatim}
rustc main.rs
\end{verbatim}
jalankan program dengan perintah berikut
\begin{verbatim}
./main
\end{verbatim}

\subsection{Cargo}
\label{sec:org4b7d541}
\subsubsection{Inisiasi Project}
\label{sec:org37593b1}
Saat kita menginstall rust biasaya akan ada bawaan cargo.

\begin{verbatim}
cargo new hello_cargo --bin
cd hello_cargo
\end{verbatim}
untuk build
\begin{verbatim}
cargo build
\end{verbatim}
setelah menjalakan cargo build, proses ini akan menghasilkan
file. Dengan menjalankan perintah dibawah ini kita akan menjalankan program
\begin{verbatim}
./target/debug/hello_cargo 
\end{verbatim}
alternatif dengan langsung mengakses file binarynya kita bisa menggunakan
perintah dibawah ini untuk langsung menjalanakan programmnya.
\begin{verbatim}
cargo run
\end{verbatim}
\end{document}
